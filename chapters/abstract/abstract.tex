\markboth{Bachelorarbeit}{Bachelorarbeit}
\addtocounter{page}{1}

\begin{flushleft}
    \begin{huge}
        \textbf{Bachelorarbeit}
    \end{huge}
    ~\\
    ~\\
    \textbf{Titel:}  Effizientes und realistisches Partikelsystem zur Simulation von Feuer und Rauch in VR-Umgebung
    ~\\
    \doublespacing
    \textbf{Gutachter:}
    \begin{description}
        \vspace{-0.2cm}
        \itemsep-8pt
        \item[–]
            Prof. Dr. Arnulph Fuhrmann (TH Köln)
        \item[–]
            Prof. Dr. rer. nat. Stefan Michael Grünvogel (TH Köln)
    \end{description}
    \vspace{-0.5cm}
    \singlespacing
    \textbf{Zusammenfassung:} Der Einsatz von Virtual Reality findet in immer mehr Bereichen 
    seinen Nutzen. Im Bereich der Brandbekämpfung könnte die Technik eine sichere und 
    kostengünstigere Alternative zu bestehenden Trainingsmethoden sein. 
    Aktuelle Anwendungen legen bisher jedoch nicht viel Wert auf wirklich immersive Erfahrungen 
    beim Rendering der Brände. Durch realistischere Renderings kann der Nutzer in echte Stresssituationen
    versetzt werden. Parallax Occlusion Mapping und Raymarching sind zwei Methoden, welche sich für die 
    Darstellung von Feuer und Rauch in Virtueller Realität anbieten.
    In dieser Arbeit werden daher zwei Shader basierend auf den vorgeschlagenen Methoden in 
    Partikelsystemen implementiert und verglichen. \singlespacing
    \textbf{Stichwörter:} Virtual Reality, Partikelsystem, Volumen Rendering, Parallax Occlusion Mapping, Echtzeitrendering\\
    \doublespacing
    \textbf{Datum:}


    \vspace{1.5cm}

    \begin{huge}
        \textbf{Bachelors Thesis}
    \end{huge}
    ~\\
    ~\\
    \textbf{Title:} Efficient and realistic particle system to render fire and smoke in VR
    ~\\
    \doublespacing
    \textbf{Reviewers:}
    \begin{description}
        \vspace{-0.2cm}
        \itemsep-8pt
        \item[–]
            Prof. Dr. Arnulph Fuhrmann (TH Köln)
        \item[–]
            Prof. Dr. rer. nat. Stefan Michael Grünvogel (TH Köln)
    \end{description}
    \vspace{-0.5cm}
    \singlespacing
    \textbf{Abstract:}
    The use of virtual reality extends into more and more areas.
    In the field of firefighting, the technique could be a safe and cheaper 
    alternative to existing fire training methods. 
    However, current applications focused less on truly immersive experiences when it 
    comes to the rendering the fires. More realistic renderings can put the user in higher
    stressful situations. 
    Parallax occlusion mapping and raymarching are two methods that are suitable 
    for displaying fire and smoke in a stereoscopic view. In this thesis, two Shaders based on these 
    methods are implemented and compared in particle systems.
    \singlespacing
    \textbf{Keywords:} Virtual Reality, Particle System, Volume Rendering, Parallax Occlusion Mapping, Real Time Rendering\\
    \doublespacing
    \textbf{Date:}
\end{flushleft}

% - Im Abstract weniger blabla \\ done
% - Beschreiben, womit das Partikelsystem erzeugt wird\\ done
% - Warum ?\\ done
% - Was wird in dieser Arbeit überhaupt gemacht? -> Vergleich Parallax Mapping / Raymarching done
