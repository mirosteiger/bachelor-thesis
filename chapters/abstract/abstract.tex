\markboth{Bachelorarbeit}{Bachelorarbeit}
\addtocounter{page}{1}

\begin{flushleft}
    \begin{huge}
        \textbf{Bachelorarbeit}
    \end{huge}
    ~\\
    ~\\
    \textbf{Titel:}  Effizientes und realistisches Partikelsystem zur Simulation von Feuer und Rauch in VR-Umgebung
    ~\\
    \doublespacing
    \textbf{Gutachter:}
    \begin{description}
        \vspace{-0.2cm}
        \itemsep-8pt
        \item[–]
            Prof. Dr. Arnulph Fuhrmann (TH Köln)
        \item[–]
            Prof. Dr. rer. nat. Stefan Michael Grünvogel (TH Köln)
    \end{description}
    \vspace{-0.5cm}
    \singlespacing
    \textbf{Zusammenfassung:} Was ein Abstract ist wird in der DIN Norm 1426 festgelegt: es ist ein Kurzreferat zur In-haltsangabe. Die Definition des American National Standards Institute (ANSI) lautet: „An abstract is defined as an abbreviated accurate representation of the con-tents of a docu-ment". Es sollten 8-10 Zeilen Text folgen.
    ~\\
    \doublespacing
    \textbf{Stichwörter:} \\
    \textbf{Datum:}

\end{flushleft}
\vspace{2cm}

\begin{flushleft}
    \begin{huge}
        \textbf{Bachelors Thesis}
    \end{huge}
    ~\\
    ~\\
    \textbf{Title:} Efficient and realistic particle system to render fire and smoke in VR
    ~\\
    \doublespacing
    \textbf{Reviewers:}
    \begin{description}
        \vspace{-0.2cm}
        \itemsep-8pt
        \item[–]
            Prof. Dr. Arnulph Fuhrmann (TH Köln)
        \item[–]
            Prof. Dr. rer. nat. Stefan Michael Grünvogel (TH Köln)
    \end{description}
    \vspace{-0.5cm}
    \singlespacing
    \textbf{Abstract:} Was ein Abstract ist wird in der DIN Norm 1426 festgelegt: es ist ein Kurzreferat zur In-haltsangabe. Die Definition des American National Standards Institute (ANSI) lautet: „An abstract is defined as an abbreviated accurate representation of the con-tents of a docu-ment". Es sollten 8-10 Zeilen Text folgen.
    ~\\
    \doublespacing
    \textbf{Keywords:} \\
    \textbf{Date:}
\end{flushleft}