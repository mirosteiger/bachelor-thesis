\markboth{Bachelorarbeit}{Bachelorarbeit}
\addtocounter{page}{1}

\begin{flushleft}
    \begin{huge}
        \textbf{Bachelorarbeit}
    \end{huge}
    ~\\
    ~\\
    \textbf{Titel:}  Effizientes und realistisches Partikelsystem zur Simulation von Feuer und Rauch in VR-Umgebung
    ~\\
    \doublespacing
    \textbf{Gutachter:}
    \begin{description}
        \vspace{-0.2cm}
        \itemsep-8pt
        \item[–]
            Prof. Dr. Arnulph Fuhrmann (TH Köln)
        \item[–]
            Prof. Dr. rer. nat. Stefan Michael Grünvogel (TH Köln)
    \end{description}
    \vspace{-0.5cm}
    \singlespacing
    \textbf{Zusammenfassung:} Der Einsatz von Virtual Reality findet in immer mehr Bereichen seinen Nutzen. Die Technik wird stetig verbessert und es gibt keine überzeugendere Möglichkeit um den Nutzer in eine andere Realität zu versetzen. Im Bereich der Brandbekämpfung könnte die Technik eine sichere und kostengünstigere Alternative zu bestehenden Trainingsmethoden sein. Im Rahmen dieser Arbeit wird ein effizientes Partikelsystem in der Game-Engine Unity entwickelt, welches eine realistische Darstellung von Feuer und Rauch in VR ermöglicht. \singlespacing
    \textbf{Stichwörter:} Virtual Reality, Partikelsystem, Volumen Rendering, Parallax Mapping, Echtzeitrendering\\
    \doublespacing
    \textbf{Datum:}


    \vspace{2cm}

    \begin{huge}
        \textbf{Bachelors Thesis}
    \end{huge}
    ~\\
    ~\\
    \textbf{Title:} Efficient and realistic particle system to render fire and smoke in VR
    ~\\
    \doublespacing
    \textbf{Reviewers:}
    \begin{description}
        \vspace{-0.2cm}
        \itemsep-8pt
        \item[–]
            Prof. Dr. Arnulph Fuhrmann (TH Köln)
        \item[–]
            Prof. Dr. rer. nat. Stefan Michael Grünvogel (TH Köln)
    \end{description}
    \vspace{-0.5cm}
    \singlespacing
    \textbf{Abstract:}
    Virtual reality is being used in more and more areas.
    The technology is constantly being improved and there is no more convincing way to put the user into another reality.
    In the field of firefighting, the technique could be a safer and cheaper alternative to existing training methods.
    As part of this work, an efficient particle system is developed in the game engine unity, which creates a realistic representation of fire and smoke in VR.
    \singlespacing
    \textbf{Keywords:} Virtual Reality, Particlesystem, Volume Rendering, Parallax Mapping, Real Time Rendering\\
    \doublespacing
    \textbf{Date:}
\end{flushleft}

% - Im Abstract weniger blabla \\
% - Beschreiben, womit das Partikelsystem erzeugt wird\\
% - Warum ?\\
% - Was wird in dieser Arbeit überhaupt gemacht? -> Vergleich Parallax Mapping / Raymarching
