\markboth{Bachelorarbeit}{Bachelorarbeit}
\addtocounter{page}{1}

\begin{flushleft}
	\begin{huge}
		\textbf{Bachelorarbeit}
	\end{huge}
	~\\
	~\\
	\textbf{Titel:}  Effizientes und realistisches Partikelsystem zur Simulation von Feuer und Rauch in VR-Umgebung
	~\\
	\doublespacing
	\textbf{Gutachter:}
	\begin{description}
		\vspace{-0.2cm}
		\itemsep-8pt
		\item[–]
			Prof. Dr. Arnulph Fuhrmann (TH Köln)
		\item[–]
			Prof. Dr. rer. nat. Stefan Michael Grünvogel (TH Köln)
	\end{description}
	\vspace{-0.4cm}
	\singlespacing
	\textbf{Zusammenfassung:} Der Einsatz von Virtual Reality findet in immer mehr Bereichen
	seinen Nutzen. Im Bereich der Brandbekämpfung könnte die Technik eine sichere und
	kostengünstigere Alternative zu bestehenden Trainingsmethoden sein.
	Aktuelle Anwendungen legen bisher jedoch nicht viel Wert auf wirklich immersive Erfahrungen
	beim Rendering der Brände. Durch realistischere Renderings kann der Nutzer in echte Stresssituationen
	versetzt werden. Parallax Occlusion Mapping und Ray Marching sind zwei Methoden, welche sich für die
	Darstellung von Feuer und Rauch in virtueller Realität anbieten könnten.
	In dieser Arbeit werden daher die beiden vorgeschlagenen Algorithmen in einem Partikelsystem implementiert, verglichen und
	hinsichtlich ihrer Performance, optischer Qualität und den Einsatzmöglichkeiten in einem Brandsimulator bewertet.
	\singlespacing
	\textbf{Stichwörter:} Virtual Reality, Partikelsystem, Parallax Occlusion Mapping, Volumen Rendering, Ray Marching, Echtzeitrendering\\
	\doublespacing
	\textbf{Datum:} 30.07.2022


	\vspace{1cm}

	\begin{huge}
		\textbf{Bachelors Thesis}
	\end{huge}
	~\\
	% ~\\
	\textbf{Title:} Efficient and realistic particle system to render fire and smoke in VR
	~\\
	\doublespacing
	\textbf{Reviewers:}
	\begin{description}
		\vspace{-0.2cm}
		\itemsep-8pt
		\item[–]
			Prof. Dr. Arnulph Fuhrmann (TH Köln)
		\item[–]
			Prof. Dr. rer. nat. Stefan Michael Grünvogel (TH Köln)
	\end{description}
	\vspace{-0.4cm}
	\singlespacing
	\textbf{Abstract:}
	The use of virtual reality extends into more and more areas.
	In context of firefighting, the technique could be a safe and cheaper
	alternative to existing fire training methods.
	However, current applications focused less on truly immersive experiences when it
	comes to the rendering of the fires. More realistic renderings can put the user in higher
	stressful situations.
	Parallax occlusion mapping and ray marching are two methods that could be suitable
	for displaying fire and smoke in a stereoscopic view.
	In this thesis, the two proposed algorithms are implemented in a particle system, compared and
	evaluated with regard to their performance, appearance and possible uses in a fire simulator.
	\singlespacing
	\textbf{Keywords:} Virtual Reality, Particle System, Parallax Occlusion Mapping, Volume Rendering, Ray Marching, Real Time Rendering\\
	\doublespacing
	\textbf{Date:} 30.07.2022
\end{flushleft}
