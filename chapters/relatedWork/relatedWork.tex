\section{Stand der Technik}

Die Idee eines Einsatztrainings für die Brandbekämpfung in Virtual Reality ist nicht neu. Es gibt mit 
'Serious Games' sogar eine eigene Kategorie, bei der es sich um Videospiele handelt, bei denen der 
Bildungsaspekt im Vordergrund steht. 
Es gibt dabei auch bei der Brandbekämpfung bereits einige Anwendungen, welche versuchen, 
sich die Möglichkeiten von VR zunutze zu machen.
Das Ziel der Anwendungen ist dabei oft das selbe. Sowohl um die Einsatzkräfte in realistischen Szenarios 
zu trainieren, ohne dabei die physische Gesundheit der Personen aufs Spiel zu setzen, als auch die 
Umwelt und die finanziellen Mittel zu schonen.



\parencite{Schlager2017} entwickelte bereits einen interaktiven Trainingssimulator für die Anwendung 
eines Feuerlöschers. Der Nutzer muss für die gegebenen Situationen aus verschiedenen Löschmitteln das 
jeweils am besten geeignete Mittel auswählen und den Brand löschen. Die Feuersimulation in dieser 
Anwendung basiert auf einem Voxelgrids und Billboard-Partikelsystemen. In Hinblick auf die Performance kam Schlager
zu dem Fazit, dass man einen Kompromiss zwischen realistischem Rendering und realitätsnahem Verhalten 
des Feuers finden muss. 


Die physikalisch korrekte Simulation von Feuer auf molekularem Level ist nach heutigem
Stand aufgrund der Komplexität der Reaktionen nur sehr begrenzt und in Echtzeitanwendungen nahezu gar nicht
möglich. Feuer wird in der Computergrafik daher mit einigen Tricks annähernd realisitsch dargestellt. 


\newpage

