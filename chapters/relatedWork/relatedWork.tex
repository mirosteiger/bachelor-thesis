\section{Related Work}
\label{sec:2}

Die Idee eines Einsatztrainings für die Brandbekämpfung in Virtual Reality ist nicht neu. Es gibt mit
'Serious Games' sogar eine eigene Kategorie, bei der es sich um Videospiele handelt, bei denen der
Bildungsaspekt im Vordergrund steht. Es gibt auch in der Brandvorbeugung und -bekämpfung bereits einige Anwendungen,
welche versuchen, sich die Möglichkeiten von VR zunutze zu machen.
Das Ziel der Anwendungen ist dabei oft das selbe. Sowohl um die Einsatzkräfte in realistischen Szenarios zu trainieren,
ohne dabei die physische Gesundheit der Personen aufs Spiel zu setzen, als auch die Umwelt und die finanziellen Mittel zu schonen.


\subsection{Partikelsysteme}
Für die Simulation von Feuer und Rauch werden in computergenerierten Welten aufgrund ihrer  Performance seit vielen Jahren 
überwiegend Partikelsysteme benutzt. 
Partikelsysteme sind ein kostengünstige Lösung, was die Rechenzeit angeht und eignen sich daher um eben solche
volumetrischen Effekte, die schwer zu modellieren und zu berechnen sind, in Echtzeitsystemen darzustellen zu können. 
Dies erzeugt auf flachen Bildschirmen die Illusion, dass diese Texturen keine flachen Bilder, 
sondern voluminös sind. Hierbei handelt es sich um Systeme von einzelnen Partikeln, welche über verschiedene 
Eigenschaften verfügen und diverse Formen annehmen können. Ein solches System kann Partikel in Form von beispielsweise Punkten, 
Linien, Sprites oder Meshes emitieren \parencite{Reeves1983}.

In \textcite{Schlager2017} wurde bereits mithilfe von Partikelsystemen ein interaktiver Trainingssimulator für die Anwendung 
eines Feuerlöschers entwickelt. Der Nutzer lernt den korrekten Umgang mit einem Feuerlöscher und muss für die gegebenen Situationen 
aus verschiedenen Löschmitteln das jeweils am besten geeignete Mittel auswählen und den Brand löschen. Solche Trainings 
gibt es zwar bereits mit echtem Brand und Feuerlöschern, jedoch lernt der Nutzer hierbei nichts darüber, wie sich ein Feuer in 
Innenräumen verhält. Auch die Rauchentwicklung im Raum wird hier nicht weiter betrachtet. Der Fokus der Arbeit liegt hier auf der 
approximiert-realistischen Simulation der Ausbreitung des Feuers, basierend auf Daten des Fire Dynamics Simulators (FDS) \parencite{FDS2004}. 
FDS ist eine auf Computational Fluid Dynamics (CFD) basierende Open-Source Software vom National Institute of Standards and Technology (NIST).
Brennbare Objekte werden durch ein Voxelgitter repräsentiert, in dem jedes Voxel Informationen wie Temperatur und Brennbarkeit beinhaltet.
Bei Schlager liegt der Fokus nicht auf dem realistischen Rendering. 
In Hinblick auf die Performance kam Schlager zu dem Fazit, dass man einen Kompromiss zwischen realistischem Rendering 
und realitätsnahem Verhalten des Feuers finden muss.


\subsection{Parallax Occlusion Mapping}
Parallax Occlusion Mapping \parencite{Brawley2004} wurde als Weiterentwicklung des Parallax Mappings \parencite{Kaneko2001} präsentiert.
POM wird mittlerweile in vielen Bereichen aufgrund der höhen Details bei gering aufgelösten Meshes \parencite{Tatarchuk2006} als 
Alternative zu Vertexdisplacements, bzw. hochauflösenden Geometrien verwendet. Es wurde mit dem Prism Parallax Occlusion Mapping
eine Methode vorgestellt um die Silhouetten der Oberflächen korrekt darstellen zu können \parencite{Dachsbacher2007}. 
Außerdem gibt es bereits Versuche POM im Zusammenhang mit Fluidsimulationen 
anzuwenden \parencite{Kufner2017}. Hierbei wurde eine 2D-Fluidsimulation, unter anderem mithilfe von POM, in eine Art Pseudo-Volumen überführt. 
Der von Kufner festgestellte Nachteil seiner Implementierung ist ein sichtbarer Treppeneffekt, welcher den Nutzen von POM für diese 
Art der Simulation einschränkt. 
%% Einsatz von POM in VR?



\subsection{Ray Marching}
Partikelsysteme, basierend auf Billboards, haben das Problem der Darstellung in VR. Alle Vorteile, die diese Art von 
Partikelsystem mit sich bringt gehen durch stereoskopisches Rendering verloren, da die Partikel plötzlich flach aussehen 
oder sich zusammen mit der Bewegung des VR-Headset drehen und kippen können. Eine weitere Methode um solche volumetrischen 
Effekte zu rendern ist das Ray-Marching. Dieses bietet neben deutlich realistischeren Renderings den Vorteil, dass dieses 
auch in der Stereoansicht gut funktioniert und überzeugende Ergebnisse liefert \parencite{Wald2006}. Die Methode bietet 
jedoch den Nachteil einer aufwändigeren Berechnung und daher auch längeren Renderingzeiten. Es wurde außerdem bereits von 
\textcite{Zhang2020} versucht, die Charakteristiken von volumetrischen Effekten auf ein Billboard-System anzuwenden. 
Hierbei hängt die Real Time-Performance allerdings stark von der Komplexität der zu rendernden Szene ab.