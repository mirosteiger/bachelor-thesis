\section{Stand der Technik}

Die Idee eines Einsatztrainings für die Brandbekämpfung in Virtual Reality ist nicht neu. Es gibt mit 
'Serious Games' sogar eine eigene Kategorie, bei der es sich um Videospiele handelt, bei denen der 
Bildungsaspekt im Vordergrund steht. 
Es gibt dabei auch bei der Brandbekämpfung bereits einige Anwendungen, welche versuchen, 
sich die Möglichkeiten von VR zunutze zu machen.
Das Ziel der Anwendungen ist dabei oft das selbe. Sowohl um die Einsatzkräfte in realistischen Szenarios 
zu trainieren, ohne dabei die physische Gesundheit der Personen aufs Spiel zu setzen, als auch die 
Umwelt und die finanziellen Mittel zu schonen.



\parencite{Schlager2017} entwickelte bereits einen interaktiven Trainingssimulator für die Anwendung 
eines Feuerlöschers. Der Nutzer muss für die gegebenen Situationen aus verschiedenen Löschmitteln das 
jeweils am besten geeignete Mittel auswählen und den Brand löschen. Die Feuersimulation in dieser 
Anwendung basiert auf einem Voxelgrids und Billboard-Partikelsystemen. In Hinblick auf die Performance kam Schlager
zu dem Fazit, dass man einen Kompromiss zwischen realistischem Rendering und realitätsnahem Verhalten 
des Feuers finden muss. 


Das erste und vorrangige Ziel des Theorieteils ist es, einen fundierten Überblick über die bisherige Forschung zu geben. Dabei geht es darum, den aktuellen Forschungsstand in Deinem Fachbereich zu erörtern und zu zeigen, welche Themen und Forschungsfragen bisher vordergründig beantwortet wurden, welche zentralen Erkenntnisse es gibt und welche Theorien und Modelle entwi- ckelt wurden. Im Theorieteil klärst Du die Grundlagen und Definitionen, das heißt Du machst klar, welche Texte und Veröffentlichungen die Grundlagenliteratur in Deinem Fachbereich darstellen, welche Fachbegriffe es gibt, wie diese definiert werden und welche großen Diskussionen oder Kontroversen es möglicherweise unter Experten gibt.


Mit Blick auf Diskussionen und Kontroversen ist Wissenschaft als permanenter Dialog zwischen Forschern auf der ganzen Welt zu verstehen. Da jede Publikation in Fachzeitschriften einen Lite- raturteil enthält, steht eine Veröffentlichung niemals „einfach so“ im Raum, ohne in die vorherige Forschung und vorherigen Er- gebnisse eingebettet zu sein. Forscher arbeiten an neuen Projek- ten und publizieren neue Studien als Antwort auf das, was bisher erforscht wurde. Sie bauen ihre Argumente auf bestehenden Theorien auf, hinterfragen Modelle und beleuchten bisher uner- forschte Teilbereiche. Dabei geht es immer darum, unbeantworte- te Fragen von Kollegen zu beantworten, ihre Ideen zu verifizieren (z.B. durch ähnliche Forschungsprojekte in anderen Ländern) oder auch Ansätze zu widerlegen (z.B. durch Datenauswertungen, die zu gegenteiligen Ergebnissen kommen). Dieser wissenschaft- liche Dialog unter Forschern funktioniert jedoch nur, wenn man den Kollegen vorher gut zugehört hat, ihre Argumente kennt und weiß, welche Fragen sie beantworten und welche sie noch stellen. Dieses „Zuhören“ ist hier ein Synonym für eine fundierte Recher- che. Du kannst Deine Abschlussarbeit nur dann einen größeren wissenschaftlichen Kontext stellen, wenn Du weißt, was bisher in diesem Bereich erforscht wurde, was zentrale Thesen sind und welche Trends es in der Forschung gibt. Bei der Verschriftlichung solltest Du versuchen, die verschiedenen Positionen aus der Lite- ratur gegenüberzustellen und miteinander zu vergleichen, sodass klar wird, in welchem Diskurs sich die Wissenschaft gerade be- findet und welche Themen diskutiert werden.


Der Literaturteil ist das Kapitel Deiner Arbeit, in dem Du Deinen Prüfern zeigen kannst, dass Du die zentralen Autoren, Theorien und Konzepte eines Themenbereichs erarbeiten kannst, diese mit-einander verknüpfen und auch einen soliden Überblick des For- schungsbereichs geben kannst.


In vielen Fachbereichen gibt es „traditionelle“ Methoden und Mo- delle, die zum Goldstandard geworden sind und immer wieder in den verschiedensten Publikationen verwendet werden (z.B. Por- ters 5 Forces oder die 4P (bzw. 7P) des Marketings in der BWL). Im Theorieteil solltest Du sowohl diese Methoden und Modelle kurz vorstellen (die in der Regel keine detailreiche Erklärung brauchen), als auch neue und noch unbekannte Ideen aus aktuel- len Publikationen. Ein Literaturüberblick kann nur dann umfas- send und vollständig sein, wenn er neben bekannten und viel zitierten Veröffentlichungen auch neuere Ideen aus Fachzeitschriften aufnimmt. So kannst Du hier zeigen, dass Du sowohl die Standardliteratur Deines Fachbereichs beherrscht als auch einen Blick für neue Trends, Themen und Methoden hast, welche die Zukunft und weitere Forschung prägen können.

Wenn Du Inhalte aus einer Quelle in Deiner Abschlussarbeit im Literaturteil beschreiben möchtest, dann sind die folgenden vier Punkte wichtig:

    Autor: Welche Einzelperson oder welches Forschungs- team hat die Ergebnisse veröffentlicht?

    Kernaussage: Was ist die zentrale Aussage bzw. das zentrale Ergebnis der Studie?

    Methodik: Mit welcher Methode wird die Forschungsfrage untersucht?

    Veröffentlichung: Wann wurden die Ergebnisse veröffent- licht (und ggf. wo)?
