\newpage
\subsection{Volume Rendering}

Unter Volume Rendering versteht man eine Reihe von Methoden, die es ermöglichen ein 3D-Datenset zu visualisieren. 
Anders als beim Rendern von Geometrien mit einer festen Oberfläche, geht es hierbei darum alle Daten aus 
dem jeweiligen Volumen darstellen zu können. Solche Volumen sind beispielsweise volumetrische Effekte wie Feuer und Rauch, Wolken oder Nebel, 
welche sich, aufgrund ihrer gasförmigen Eigenschaften, nicht wirklich realistisch mit Geometrie darstellen lassen. 
Diese Volumen bestehen im Fall von Rauch aus Millionen winzigen Partikeln. 
Möchte man also Volumendaten rendern, so geraten die vorherigen Herangehensweisen schnell an ihre Grenzen.
Beim Grundgedanken des Volume Renderings geht es daher um die Frage, wie diese vielen kleinen Partikel mit einfallendem Licht interagieren
und wie man die darstellen kann. 
Es gibt einige Faktoren, die beeinflussen was das Auge am Ende von diesem Volumen sieht. Wenn das Licht durch solch ein Volumen wandert 
kann es absorbiert, reflektiert oder gestreut werden. 
Um die Durchlässigkeit des Volumens zu berechnen, kann mithilfe des Lambert-beerschen Gesetzes berechnet werden, wie viel Licht beim 
Durchlaufen des Volumens absorbiert wird. 

\scalebox{1.3}{%
$$\textstyle I_{out} = I_{in} \cdot e^{- ( \alpha \cdot \epsilon \cdot d )}$$
$}



\subsubsection{Ray Marching}

% \subsubsection{Texturbasierte Volumen}