\subsection{Feuer- und Rauchsimulationen}

\subsubsection{Eigenschaften}
Chemischen Reaktionen wie Feuer, oder der aus winzigen Partikeln bestehende Rauch,
lassen sich nur schwer physikalisch exakt simulieren. Es haben sich hierfür verschiedene Ansätze entwickelt, um ein
annähernd realistisches Rendering dieser Phänomene (mit volumetrischem Charakter) zu erschaffen. 
Es gibt hierbei beispielsweise die physikalisch basierten Ansätze. 
Diese beruhen auf aufwändig berechneten Fluidsimulationen und werden vorallem in der Gefahrensimulation verwendet.
Diese sind aufgrund ihrer Komplexität aber nur bedingt echtzeitfähig und werden meist zuvor berechnet. 
Der Fokus liegt dabei meist auf der realistischen Ausbreitung des Feuers und des Rauchs, um beispielsweise Fluchtwege 
zu testen oder Evakuierungsszenarien zu simulieren. Diese Anwendungsfälle müssen nicht in Echtzeit berechnet werden.
Daher ist diese Art der Simulation auch für Anwendungen, in denen es um die Interaktion mit dem Feuer geht, nicht 
effizient nutzbar. Ein Beispiel dafür ist der Einsatz in einem Simulator, der den Umgang mit dem Feuerlöscher
näher bringen soll. Diese basieren zumeist auf effizienteren Partikelsystemen, mit denen sich solche Phänomene 
durch eine Vielzahl an Parametern ebenfalls darstellen lassen. Dieser Ansatz ist dabei eher von artistischer Natur. 
Zwar lassen sich die Partikel – beispielsweise mit Vektorfeldern – genau steuern, um ein realistisches Aussehen zu 
erzeugen gibt es aber vorallem in Videospielen andere Methoden, die sich durchgesetzt haben. 

Feuer, bzw. Flammen sind brennende, Licht und Wärme emittierende Gase, erzeugt durch eine chemische Reaktion. 
\begin{figure}[h!]
	\includegraphics[width=0.49\textwidth]{Grafiken/Basics/Fire/explosion_0000.png}
	\includegraphics[width=0.49\textwidth]{Grafiken/Basics/Fire/explosion2_0000_0000.png}
	\centering
	\begin{footnotesize}
		\caption{Rendering von Explosionen. Simuliert in EmberGen von JangaFX }

		\label{fig:explosion}
	\end{footnotesize}
\end{figure}


\subsubsection{Partikelsysteme}
<<<<TODO>>>>
