\subsection{Virtual Reality}
\subsubsection{Konzept}

Hinter dem Begriff Virtual Reality (VR) verbirgt sich das Konzept einer
künstlichen, von Computern generierten Welt. Der Nutzer kann in diese Welt
eintauchen und hat dabei die Möglichkeit sich als Betrachter in dieser Welt
umzuschauen oder sogar mit dieser Welt zu interagieren.
Dabei gibt es verschiedene Arten von VR. Zum einen die "Non-immersive Virtual Reality".
Hierbei steuert der Nutzer seine virtuelle Umgebung, ist sich dabei aber noch bewusst
in welcher Welt er sich tatsächlich befindet. Die Interaktion geschieht üblicherweise durch
Eingabegeräte wie Controller, Maus oder Tastatur. Ein weit verbreitetes Anwendungsgebiet sind
dabei herkömmliche Videospiele.
Gegenüberstehend gibt es dagegen die "Fully Immersive Virtual Reality". Hierbei wird der Nutzer durch
spezielle Hardware, zum Beispiel mithilfe eines Head-Mounted-Displays (HMD), einem sogenannten VR-Headset,
selbst in eine virtuelle dreidimensionale Umgebung versetzt.
Durch visuelles, auditives und teilweise auch haptisches Feedback kann der Nutzer dabei immer weiter
in die virtuelle Welt eintauchen.
Auch hier verfügt der Nutzer über Eingabegeräte wie dem Headset, Controllern oder Laufbändern. Diese sind
jedoch in der Benutzung näher an der bekannten Realität. So kann sich der Nutzer zb. mit einer Kopfbewegung
in der virtuellen Welt umsehen oder Dinge anfassen und mit diesen interagieren. Dieser Einfluss auf die
Umgebung sorgt dafür, dass sich eine Simulation echter anfühlen kann.


\subsubsection{Immersion}
\subsubsection{Tiefenwahrnehmung}

% \subsubsection{Head-Mounted-Display}
% Die Idee von Fully Immersive Virtual Realities baut darauf auf, die Sinne des Nutzers so überzeugend zu täuschen,
% sodass dieser glaubt, er befinde sich in einer anderen Welt. Die nächste Stufe nach Immersion ist die
% Präsenz. Präsenz beschreibt hierbei das Gefühl, bzw. die Illusion, dass sich der Nutzer tatsächlich
% physisch in dieser computergenerierten Welt befindet. \parencite{Doerner2022}

% Der Mensch ist ein visuell orientiertes Lebewesen. Daher ist der Einsatz einer VR-Brille 
% der wichtigste Faktor um eine solche Illusion zu erzeugen. Der Eindruck von Raum und Tiefe wird
% dabei vom Gehirn erzeugt. Die Information dazu kommen von den beiden Augen. Die Tiefe wird mithilfe 
% binokularer Disparität erzeugt. Dieser Begriff bezeichnet grob gesagt den kleinen, aber bedeutenden
% Unterschied zwischen den beiden einzelnen Bildern, welche von den Augen erzeugt werden. Daraus kann das Gehirn
% abschätzen, wie weit ein Objekt entfernt ist. \parencite{Tauer2010}


% Der Einsatz einer VR-Brille bringt einige technische Anforderungen mit sich, welche sich deutlich
% von denen eines herkömmlichen Monitors unterscheiden. Der Grund, warum diese Brillen funktionieren und eine
% so überzeugende Wahrnehmung der 3D-Welt um sich herum
% Eine solche Brille basiert auf zwei Displays, welche
% sich direkt vor den Augen des Nutzers befinden. \textcolor{red}{GRAFIK VON STEREOVIEW}
