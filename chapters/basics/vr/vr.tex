\subsection{Virtual Reality}
\subsubsection{Konzept}

Hinter dem Begriff Virtual Reality (VR) verbirgt sich das Konzept einer
künstlichen, von Computern generierten Welt.
Der Nutzer kann in diese Welt eintauchen und hat dabei die Möglichkeit, sich als Betrachter in dieser Welt
umzuschauen, oder sogar mit dieser Welt zu interagieren. Das erste Konzept eines VR-Headsets mit Kopftracking
wurde bereits in den 60er Jahren von Ivan Sutherland entworfen. \parencite{Sutherland1965, Sutherland1968}

Heutzutage gibt es verschiedene Arten von VR. Zum einen die "Non-immersive Virtual Reality".
Hierbei steuert der Nutzer seine virtuelle Umgebung, ist sich dabei aber noch bewusst,
in welcher Realität er sich tatsächlich befindet. Die Interaktion geschieht üblicherweise durch
Eingabegeräte wie Controller, Maus oder Tastatur. Ein weit verbreitetes Anwendungsgebiet sind
dabei herkömmliche Videospiele. Gegenüberstehend gibt es dagegen die "Fully Immersive Virtual Reality".
Hierbei wird der Nutzer durch spezielle Hardware, zum Beispiel mithilfe eines Head-Mounted-Displays (HMD),
einem sogenannten VR-Headset, selbst in eine virtuelle dreidimensionale Umgebung versetzt.
Durch visuelles, auditives und teilweise auch haptisches Feedback kann der Nutzer dabei immer weiter
in die virtuelle Welt eintauchen.
Auch hier verfügt der Nutzer über spezielle Eingabegeräte wie dem Headset, Controllern oder Laufbändern. Diese sind
jedoch in ihrer Benutzung näher an der bekannten Realität. So kann sich der Nutzer z.B. mit einer Kopfbewegung
in der virtuellen Welt umsehen oder Dinge anfassen und mit diesen interagieren. Dieser Einfluss auf die
Umgebung sorgt dafür, dass sich eine Simulation echter anfühlen kann.

Die Idee von Fully Immersive Virtual Realities baut dabei darauf auf, die Sinne des Nutzers so überzeugend zu täuschen,
sodass dieser glaubt, er befinde sich in einer anderen Welt. Die nächste Stufe nach Immersion ist die
Präsenz. Präsenz beschreibt hierbei das Gefühl, bzw. die Illusion, dass sich der Nutzer tatsächlich
physisch in dieser computergenerierten Welt befindet und diese nicht mehr von seiner wirklichen Realität unterscheiden kann
\parencite{Schuemie2001}.



\subsubsection{Tiefenwahrnehmung}


Der Mensch ist ein visuell orientiertes Lebewesen. Daher ist der Einsatz einer VR-Brille einer der wichtigsten
Faktoren, um eine solche Illusion zu erzeugen. Der Eindruck, sich in einer anderen dreidimensionalen Welt zu befinden, 
wird von der Illusion von Raum und Tiefe vom Gehirn erzeugt.
Die Information dazu werden aus den Bildern beider Augen generiert.
Das Gehirn hat einige Möglichkeiten, sich ein Verständnis von Räumlichkeit zu schaffen.
Die Tiefenwahrnehmung wird aufgrund binokularer Disparität erzeugt. Dieser Begriff bezeichnet grob gesagt den kleinen, aber bedeutenden
Unterschied zwischen den beiden einzelnen Bildern, welche von den Augen erzeugt werden. Daraus kann das Gehirn in etwa
abschätzen, wie weit ein Objekt entfernt ist.
Diese Disparitäten entstehen durch Informationen wie Verdeckung, Schattenwurf oder die unterschiedlichen Orientierungen von Linien zwischen
beiden Blickwinkeln \parencite{Tauer2010}. Mit Hilfe aller dieser Informationen entsteht ein Eindruck von räumlicher Tiefe, 
jedoch ist es noch nicht ganz möglich, mit den Methoden eine wirklich genaue Einschätzung der Entfernungen in dieser Welt zu erhalten. 
Gerade transparente Objekte lassen sich nicht genau verorten \parencite{ElJamiy2019}.



\begin{figure}[!h]
	\centering
	\adjincludegraphics[width=0.49\textwidth, trim={.3\width} {.4\height} {.3\width} {.1\height},clip]{Grafiken/Basics/VR/Stereo_Left.png}
	\adjincludegraphics[width=0.49\textwidth, trim={.3\width} {.4\height} {.3\width} {.1\height},clip]{Grafiken/Basics/VR/Stereo_Right.png}
	\begin{footnotesize}
		\caption{Perspektive des linken, sowie des rechten Auges. Die Anzahl der sichtbaren Punkte auf den Seiten des Würfels verdeutlicht die leicht verschiedenen Blickwinkel der Augen.}
	\end{footnotesize}
\end{figure}


\subsubsection{Head-Mounted-Display}
Die meisten kommerziellen Systeme basieren heutzutage auf der Nutzung eines HMD.
Diese können sowohl Bild, als auch Ton ausgeben. Für diese Arbeit ist jedoch nur die visuelle Komponente interessant.
Ein HMD basiert auf zwei Displays, welche sich direkt vor den Augen des Nutzers befinden \parencite{Sutherland1968}.
Diese Displays machen sich die Eigenschaften des menschlichen Sehens zunutze, um die Illusion von Tiefe hervorzurufen.
Auf jedem Display wird jeweils ein Bild gerendert, welches aus leicht verschobenen Positionen heraus berechnet wird.
Dabei werden in der Software, anstatt der üblichen einzelnen Kamera für das Rendering, die Bilder von zwei virtuellen Kameras aufgenommen,
welche die Abstände der beiden Augen simulieren \parencite{NVIDIA2010}. Dieser Abstand beträgt den durchschnittlichen
Augenabstand eines Menschen. Normalerweise liegt dieser bei ca. 65mm.

Der Einsatz einer VR-Brille bringt einige technische Anforderungen mit sich, welche sich deutlich
von denen eines herkömmlichen Monitors unterscheiden. Die empfohlenen Spezifikationen unterscheiden sich dabei je nach Art und
Hersteller der Brille. Für die Implementierung und das Testing der Methoden dieser Arbeit wurde die HP Reverb G2 mit den folgenden
Spezifikationen verwendet:


\begin{table}[!h]
	\renewcommand*{\arraystretch}{2}
	\setlength{\tabcolsep}{1.5cm}
	\begin{tabular}{lll}
		\hspace{-1.5cm}Bildschirm          & 2 x 2,89-Zoll-LCD                    \\ \hline
		\hspace{-1.5cm}Auflösung           & \makecell[l]{2160 x 2160 pro Auge    \\4320 x 2160 kombiniert}  \\ \hline
		\hspace{-1.5cm}Field OF View       & \raisebox{-0.6ex}{\~{ }}114°         \\ \hline
		\hspace{-1.5cm}Bildrate            & 90Hz                                 \\ \hline
		\hspace{-1.5cm}Trackingarchitektur & 6DoF                                 \\ \hline
		\hspace{-1.5cm}Augenabstand        & 64 mm +/- 4 mm durch Hardware Slider \\ \hline
	\end{tabular}
	\caption[Tabelle 1]{HP Reverb G2 Headset Spezifikationen \parencite{HPG2}}
\end{table}








