\subsection{Texture Mapping}

Texture Mapping bezeichnet ein Shading-Verfahren, welches zweidimensionale Texturen auf ein dreidimensionales Objekt
abbildet. Um die flachen Texturen auf die Oberfläche des Meshs abbilden zu können, muss das Objekt 'UV-unwrapped' werden.
Durch diesen Vorgang wird jedem Punkt auf der Oberfläche des Meshs ein Punkt auf der Textur zugewiesen \parencite{Catmull1974} \parencite{Blinn1976}.
Ein sehr einfaches Beispiel zur Veranschaulichung ist dabei das Würfelnetz.
\textcolor{red}{>>GRAFIK VON WÜRFELNETZ/UV-UNWRAPPING<<}

Texture Mapping sorgt dafür die Objekte 'anzumalen'.
Reale Objekte haben oft sehr detaillierte Oberflächeneigenschaften und sind eigentlich niemals wirklich glatt.
Geometrische Unebenheiten und Feinheiten, wie Kratzer, Rillen und Schmutz lassen sich zwar mithilfe von
Texturen andeuten, jedoch bleibt die Oberfläche komplett glatt. Diese rauhen Oberflächen zu modellieren resultiert
aber in einer deutlich höheren Polygon-Anzahl, was die Performance in großen Szenen schnell negativ beeinflussen kann.
Daher wurden Mapping-Verfahren als Ergänzung entwickelt um die virtuelle Auflösung
solch komplexer Oberflächen kostengünstig zu erhöhen, ohne dabei die Komplexität der Geometrie zu verändern.
Dabei gibt es verschiedenste Shader, welche mithilfe weiterer, spezieller Texturen eine deutlich detailliertere
Oberfläche simulieren können.


% \subsubsection{Normal Mapping}
\subsubsection{Bump Mapping}

Eine Art die Oberflächen zu rendern nennt sich Bump Mapping.
Hierbei werden mithilfe von Bump- oder Normalmaps Details generiert, welche den Eindruck einer realistischen
Struktur der Oberfläche erzeugen, obwohl die Geometrie an sich keinerlei Informationen dazu beinhaltet \parencite{Blinn1978}.

Bump Maps sind Texturen, basierend auf schwarz und weiß, bei denen die Helligkeit eines Pixels einen Höhenwert
repräsentiert. Normal Maps sind eine bessere Variante der Bump Maps.
Hier werden die Richtungen der Normalen in jedem Pixel durch einen Vektor repräsentiert, welcher sich aus den
RGB-Werten eines jeden Pixels ergibt. 
Der Eindruck der Unebenheiten wird durch Schattenwurf erzeugt. Dieser wird aus Lichtquellen, 
deren Einfallsrichtung und den Normalen aus der Textur berechnet. Somit wird auch bei geringer Polygonanzahl eine
deutlich realistischer aussehende Oberfläche gerendert. 
\parencite{Cohen1998}.

Shader basierend auf diesen Methoden sind dabei aber stark blickwinkelabhängig.
Von vorne betrachtet funktioniert die Illusion, je spitzer jedoch der Winkel zwischen Betrachter und Textur
wird, desto auffälliger wird die Tatsache, dass die Silhouette des Objekts immer noch flach ist, 
da die Geometrie hierbei nicht verändert wird. 

% \subsubsection{Displacement Mapping}

% Benötigt Heightmap


% Tatsächliche Modifizierung der Vertices entlang ihrer Normalen


% Da Positionen der Vertices modifiziert werden -> höher aufgelöstes Mesh notwendig.


\subsubsection{Parallax Mapping}

Parallax Mapping ist eine Methode, um sich die Möglichkeiten der Bump Maps zu nutze zu machen, aber ähnliche Effekte
wie bei der Nutzung von Displacement Maps zu erzielen. Anders als bei tatsächlicher Modifizierung der Vertices 
durch Displacement Maps werden hier nur die Texturkoordinaten abhängig vom Blickwinkel verschoben. \parencite{Kaneko2001}
Durch Bewegung der Oberfläche entsteht somit ein realistischerer Eindruck von Tiefe in der Textur. Dabei ist Parallax Mapping
allerdings immer noch effizienter als Vertex-Displacement. Parallax Mapping hat jedoch auch den Nachteil 
diverser optischer Artefakte


\subsubsection{Parallax Occlusion Mapping}

Parallax Occlusion Mapping (POM) ist eine verbesserte Variante des Parallax Mapping. 
\parencite{Tatarchuk2006}




% TODO:
% - Displacement Mapping kurz beschreiben
% - Parallax Mapping Nachteile
% - POM runterschreiben