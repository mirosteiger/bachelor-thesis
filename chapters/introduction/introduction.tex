\section{Einleitung}
\markboth{Einleitung}{Einleitung}
\noindent

Die Simulation von Feuer und Rauch ist ein viel diskutiertes Thema in der Computergrafik. Einerseits
spielen diese besonders in Videospielen und Filmproduktionen eine außerordentlich große Rolle, auf der
anderen Seite helfen solche Simulationen auch im Bereich der Gefahrenbekämpfung und -vorbeugung.
So kann eine realistische Feuersimulation dazu beitragen, einen Trainingssimulator für die Feuerwehr zu
entwerfen \parencite{Schlager2017}. Der Nutzer kann dabei mithilfe von Head-Mounted-Displays in ein virtuelles Brandszenario versetzt werden,
ohne dabei wirklichen Gefahren ausgesetzt zu sein. Eine solche Anwendung findet seinen Nutzen sowohl im Training
von Einsatzkräften, als auch bereits bei der Entscheidung, einem solchen Beruf nachzugehen.
Hierbei gibt es verschiedenste Ansätze, um ein künstliches erzeugtes Feuer auf einem
Bildschirm anzeigen zu lassen. Eine gängige Methode für das Rendering von Gasen und Flüssigkeiten
in Videospielen, in denen sich auch Feuer und Rauch aufgrund ihrer physikalischen Eigenschaften
wiederfinden, sind der Einsatz von Partikelsystemen. Die physikalisch korrekte Simulation kann dabei,
unter anderem auf Basis von Fluidsimulationen, sehr realitätsnah dargestellt werden.
Auf realen, physikalischen Eigenschaften basierende Simulationen sind jedoch sehr aufwändig in der
Berechnung und bisher kaum für die Echtzeitanwendung gedacht.
Gerade in Virtual-Reality-Systemen, in denen die Performance besonders wichtig für das Nutzererlebnis sind,
eignet sich die aufwändige Simulation von Fluiden aufgrund ihrer Performance nicht. Als Alternative hat
sich hierfür eine Art von Partikelsystem etabliert, welche sich anstatt der physikalisch korrekten
Eigenschaften eher an einer optischen Illusion unter Anwendung animierter Texturen bedienen.
Hierbei ist das Konzept des 'Billboardings' ein weit verbreiteter und beliebter Ansatz,
um realistischere Renderings der Partikel zu erzeugen. Diese bieten eine optisch überzeugende und
dabei noch effiziente Lösung.



\subsection{Problem}

% Problem muss hier noch beschrieben werden: 
%   - Warum sind realistische Simulationen sinnvoll/wichtig?
%   - Warum funktioniert das in VR nicht?
%   - Was heißt das für die Immersion? -> Feuer sieht unecht und kacke aus, 
%     keine Stresssituation, bzw. Gefühl von Gefahr o.Ä.

Ein solches Partikelsystem, basierend auf Texturen, kann auf einem flachen Bildschirm realistisch
und optisch überzeugend aussehen. In einer VR-Umgebung gerät diese Methode jedoch leider an seine Grenzen.
Die Illusion basiert auf der Eigenschaft, dass die Normalen der flachen Partikel immer zum Betrachter, bzw. der Kamera
orientiert sind. Dadurch lässt sich nicht erkennen, dass lediglich flache Texturen zum Einsatz kommen.
Aufgrund dieser Eigenschaft sehen die Partikel voluminös aus und täuschen Tiefe vor.
In VR-Anwendungen müssen jedoch immer zwei Bilder erzeugt werden, eins für jedes Auge.
Dadurch, dass sich die flachen Billboards immer zur jeweiligen Kamera, bzw. zum Mittelpunkt
zwischen beiden Augen orientieren, zerstört dies die Illusion, und es lässt sich erkennen, dass
die Partikel ihre Tiefe lediglich vortäuschen. So sieht ein Feuer-Partikelsystem, basierend auf Texturen,
schnell sehr unecht aus und die Immersion ist gestört.
Um in einem Traingsszenario der Feuerwehr jedoch einen wirklichen Nutzen zu finden, sollte das Feuer
so realistisch wie möglich aussehen. Erst dann wird der Nutzer in eine echte Stresssituation versetzt
und kann sich somit besser auf einen Einsatz in der realen Welt vorbereiten.

\subsection{Zielsetzung}

Ziel dieser Arbeit ist es, zwei Methoden für das Rendering von Feuer und Rauch zu implementieren und zu vergleichen, 
inwiefern sich damit auch in Virtual Reality-Umgebungen ein realistisches Bild dieser Phänomene erzeugen lassen kann. 
Durch eine bessere, plausible Darstellung in der 
Stereo-Ansicht kann beim Nutzer ein realeres Gefühl von Gefahr hervorgerufen werden, welches den Trainingseffekt 
deutlich erhöhen kann. Dazu sollen beide Varianten in Kombination mit Partikelsystemen in Unity entworfen werden, 
welche sowohl in der Stereoansicht funktionieren, als auch in Hinblick auf die benötigte Performance für VR-Renderings die 
Mindest-Framerate einhalten können. 


\subsection{Struktur der Arbeit}

Zunächst wird in Kapitel 2: \textbf{\nameref{sec:2}} ein kurzer Überblick über bereits vorhandene, verwandte Arbeiten
und Beiträge aus den Bereichen Partikelsystemen, Parallax Occlusion Mapping, Ray Marching und deren Anwendung in VR gegeben.
Das Kapitel 3: \textbf{\nameref{sec:3}} setzt sich mit den thematischen Grundlagen zu VR, Feuer- und Rauchsimulationen,
Mapping Verfahren und Volume Rendering auseinander, welche benötigt werden, um die geplanten Partikelsysteme zu implementieren.
Im vierten Teil der Arbeit geht es um die \textbf{\nameref{sec:4}} der gewählten Methoden. Hierbei werden die Schritte 
von der Idee über die Erstellung der Texturen und Shader bis hin zu fertigen Partikelsystemen beschrieben.
Anschließend werden im 5. Kapitel \textbf{\nameref{sec:5}} beide  Methoden individuell auf auf ihre Vor- und Nachteile hinsichtlich 
der Performance und ihrer optischen Qualität geprüft. 
Im letzten Teil wird in Kapitel 6: \textbf{\nameref{sec:6}} eine Zusammenfassung der Ergebnisse, sowie mögliche Anwendungsszenarien
der Methoden diskutiert. Abschließen werden einige weitere Optimierungsmöglichkeiten vorgestellt. 



