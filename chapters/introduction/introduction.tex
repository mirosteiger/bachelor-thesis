\section{Einleitung}
\markboth{Einleitung}{Einleitung}
\noindent
Die Simulation von Feuer und Rauch ist ein viel diskutiertes Thema in der Computergrafik.
Dabei gibt es verschiedene Ansätze.
Eine gängige Methode für das Rendering von Gasen, Flüssigkeiten oder Haaren in Videospielen, sind der 
Einsatz von Partikelsystemen. Die physikalisch korrekte Simulation kann dabei, unter anderem mithilfe von 
Fluidsimulationen, sehr realitätsnah dargestellt werden.
Simulationen, basierend auf realen physikalischen Eigenschaften, sind jedoch sehr aufwändig in der Berechnung
und kaum für die Echtzeitanwendung gedacht.
Gerade in Virtual-Reality-Systemen, in denen die Performance extrem wichtig für das Nutzererlebnis sind,
eignet sich die aufwändige Simulation von Fluiden nicht. Als Alternative haben sich hierfür eine Art von
Partikelsysteme etabliert, welche sich anstatt der physikalisch korrekten Eigenschaften eher an einer 
optischen Illusion mithilfe animierte Texturen bedienen. Hierbei ist das Konzept der Billboards ein weit 
verbreiteter und beliebter Ansatz, um realistischere Renderings der Partikel zu erzeugen. 
Diese bieten eine optisch überzeugende und dabei noch hocheffiziente Lösung.

%Problem muss hier noch beschrieben werden: 
%   - Warum sind realistische Simulationen sinnvoll/wichtig?
%   - Warum funktioniert das in VR nicht?
%   - Was heißt das für die Immersion? -> Feuer sieht unecht und kacke aus, keine Stresssituation, bzw. Gefühl
%     von Gefahr o.Ä.


\subsection{Zielsetzung}

Ziel dieser Arbeit ist es, ein Partikelsystem zu entwickeln, welches alle Vorteile der Billboard-Technik 
nutzen kann um auch in Virtual Reality ein realistisches Bild von Feuer und Rauch erzeugen kann. Durch eine 
bessere, plausible Darstellung in der Stereo-Ansicht kann beim Nutzer ein realeres Gefühl von Gefahr 
hervorgerufen werden. 

\subsection{Struktur der Arbeit}


\newpage
