\section{Evaluierung der Methoden}
\label{sec:5}
\markboth{Evaluierung der Methoden}{Evaluierung der Methoden}

Beide Methoden haben ihre Vor- und Nachteile. Die realistische Darstellung von voluminösen Materialien verlangt 
eine besondere Herangehensweise, welche sich – im Gegensatz zu undurchsichtigen, festen Oberflächen – 
nicht überzeugend mit herkömmlichen Texture-Mapping Methoden umsetzen lässt. 
Die Implementierung der Systeme ist aufgrund der begrenzten Zeit im Rahmen dieser Arbeit nur prototypisch mithilfe der 
zur Verfügung stehenden Funktionen aus Unity implementiert und daher nicht bestmöglich optimiert. 
In \textbf{\autoref{sec:6.3}} werden einige Optimierungsmöglichkeiten präsentiert. 
Die Vor- und Nachteile, sowie das Potential für den Nutzen der jeweiligen Methoden lassen sich dennoch aus dem Entwurf 
herausarbeiten und werden im Folgenden beschrieben . 


\subsection{Parallax Occlusion Mapping}
\label{sec:5.1}

PRO: \newline
- Fluidsimulation -> geiler Look\newline
- Beleuchtung ist recht einfach zu machen \newline
- Beleuchtung ist dynamisch für eine geringe Zahl an Lichtquellen
- Für dunklen Rauch ist die genaue Berechnung von Schatten nicht so wichtig
- In Texturen lassen sich noch mehr Infos speichern (Color-Key (für Feuer), Emission, Gradients, etc.) \newline
- Geringe Shaderlaufzeit\newline
- Unendlich wiederholbar\newline
- Sieht von weiter weg ganz gut aus\newline
- Offset wird für beide Augen individuell berechnet\newline
- Startframe kann zufällig gewählt werden -> gibt dem Rauch ein bisschen Variation ohne Overhead


POM ist eine komplexere Variante des Normalmappings. Dementsprechend ist auch POM eine Technik für feste Oberflächen und weniger
für den Einsatz bei der Simulation von Volumen gedacht. Um aber die Effizienz der billboard-basierten Partikelsysteme 
nutzen zu können ist das Parallax Occlusion Mapping eine interessante Möglichkeit. Trotz der erhöhten Komplexität des Algorithmus
gegenüber klassischeren Mappingverfahren läuft die Anwendung auf dem Testsystem stabil mit 90FPS. Hier gibt es also keine 
Bedenken hinsichtlich der Shaderperformance. Auch die aufwändige Berechnung der Fluidsimulationen wird hier im Voraus erledigt. Somit
ist es möglich realistisch aussehende Partikelsysteme unter geringem Rechenaufwand in der Echtzeitanwendung zu erzeugen.
 

CONTRA\newline
- Transparenz ist irgendwie weird\newline
- Geringe Auflösung der einzelnen Frames \newline
 -> Houdini Non-Commercial-Version erlaubt keine höhere Auflösung\newline
- Basiert immernoch auf Billboards\newline
 -> Trotz simulierter Tiefe ein flacher Eindruck in VR\newline
- Funktioniert nicht gut bei mehreren Lichtquellen\newline
- Schattenwurf auf Umgebung funktioniert nicht\newline
- Beleuchtung der  Umgebung auch nicht möglich\newline




\subsection{Ray Marching}
\label{sec:5.2}