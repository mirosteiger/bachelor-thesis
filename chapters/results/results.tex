\section{Ergebnisse}
\label{sec:6}
Hier kommen die Erkenntnisse meiner Arbeit rein
\subsection{Fazit}
\label{sec:6.1}
Das Ziel dieser Bachelorarbeit war es zwei vielversprechende Methoden für das Stereorendering von Feuer und Rauch testweise zu implementieren, 
zu vergleichen und die Eignung den möglichen Einsatz in VR-Systemen zu beurteilen. Nach Evaluierung der Ergebnisse von Parallax Occlusion Mapping
und Ray Marching stellt sich heraus, dass sich POM potentiell für das Rendering von Rauch geeignet sein könnte, die Standard-Variante wie sie im Rahmen 
dieser Arbeit für die Betrachtung in VR verwendet wurde jedoch einige Probleme mit sich bringt. Parallax Occlusion Mapping bringt den großen Vorteil
der Berechnung des Offsets und der Verdeckung basierend auf dem Blickwinkel mit sich, was sich besonders gut für den Nutzen von HMDs eignet. Jedoch 



\subsection{Limitationen}
\label{sec:6.2}

- Unity URP: Komplexe Renderingpipeline, Dokumentation nicht immer gut, viel Wissen aus Foren
- Houdini: Non-Commercial-Version erlaubt nur eine gewisse Qualität der Exporte, Alpha Channel kaputt und damit useless.
- Heightmap für Smoke nicht exportierbar
- Breit gefächertes Feld mit vielen Varianten und unendlich Optimierungspotential -> Man kann nicht alles schaffen





\subsection{Ausblick}
\label{sec:6.3}
- Viele Möglichkeiten zur Optimierung
- Animationen können sehr aufwändig und damit deutlich realistischer werden
- Möglichkeiten die Animation als Volumentextur zu rendern (4D-Texturen?)
- 

\newpage
