\section{Schluss}
\label{sec:6}

\subsection{Fazit}
\label{sec:6.1}
Das Ziel dieser Bachelorarbeit war es zwei vielversprechende Methoden für das Stereorendering von Feuer und Rauch testweise zu implementieren, 
zu vergleichen und die Eignung für den möglichen Einsatz in VR-Systemen zu beurteilen. Nach Evaluierung der Ergebnisse von Parallax Occlusion Mapping
und Ray Marching stellt sich heraus, dass POM potentiell für das Rendering von Rauch geeignet sein könnte, die Standard-Variante wie sie im Rahmen 
dieser Arbeit in Kombination mit den Texture Sheets für die Betrachtung in VR verwendet wurde jedoch einige Probleme mit sich bringt. Parallax Occlusion 
Mapping bringt den großen Vorteil der Berechnung des Offsets und der Verdeckung basierend auf dem Blickwinkel mit sich, was sich besonders gut für den 
Einsatz von HMDs eignet. Jedoch ist aus den Erkenntnissen dieser Arbeit der Volume Rendering-Ansatz vorzuziehen. 
Durch das Rendering eines tatsächlichen Volumens sieht der Rauch in Virtueller Realität realistischer asu, als durch eine Illusion auf einem zweidimensionalen Sprite. 



\subsection{Limitationen}
\label{sec:6.2}

Die Erkenntnisse dieser Arbeit beziehen sich auf sehr grundlegende Implementierungen der beiden Varianten als Proof of Concept. Dies hängt zum einen mit 
Unitys Universal Rendering Pipeline zusammen, welche einerseits vorteilhaft für VR-Anwendungen ist, es andererseits aber beispielsweise schwieriger macht
auf die Lichter in der Szene zuzugreifen. Zudem ist die Dokumentation der Software, gerade was die Programmierung von Shadern angeht, teilweise nicht hilfreich. 
Es musste viel auf Ideen und Meinungen in Foren oder Videos zurückgegriffen werden um beide Methoden im zeitlichen Rahmen implementieren zu können. 
Weiterhin wurde für die Erstellung der Lightmaps zunächst auf eine andere Software, Houdini von SideFX zurückgegriffen.
Houdini bietet eine Non-Commercial-Version an, die einige Limitationen mit sich bringt. Darunter beispielsweise eine begrenzte Auflösung der Texturen, 
teilweise fehlende Informationen beim rendern der Bildsequenzen und Wasserzeichen in den Texture Sheets. Die Erstellung und Aufbereitung der Texturen hat viel Zeit
in Anspruch genommen, wobei das Ergebnis am Ende qualitativ fast nicht nutzbar, sodass die Software gewechselt werden musste um diesen Ansatz weiter verfolgen zu können.
Mit mehr Zeit und Ressourcen lassen sich hier jedoch brauchbare Ergebnisse erzeugen, die letztendlich die Qualität des Partikelsystems erhöhen können. 



\subsection{Ausblick}
\label{sec:6.3}
Diese Arbeit zeigt auf, dass die beiden Methoden für das Gebiet der Feuer und Rauchsimulation für virtuelle Umgebungen Potential haben, allerdings auch noch 
weiterentwickelt werden müssen um Einsatzfähig zu werden.   
Die Simulation von Phänomenen wie Feuer und Rauch ein breit gefächertes Feld. Es gibt unzählige Optimierungsmöglichkeiten, was es schwerer macht, einen Fokus auf
bestimmte Dinge zu setzen. Viele Themen die ein Feuer realistischer machen, wie beispielsweise die korrekte Beleuchtung der Umgebung oder das korrekte Verhalten 
der Partikelsysteme konnten in dieser Arbeit aufgrund der Zeit und ihrer Komplexität nicht angegangen werden und könnten für weitere Forschungsansätze interessant sein. 
Zudem könnte mehr Variation und damit Realismus durch verschiedene Animationen erzeugt werden. Gerade im Bereich der prozeduralen Texturgenerierung könnte es interessant
sein eine Art 4D-Textur zu entwickeln.

\newpage
