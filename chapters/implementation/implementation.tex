\section{Konzeption und Umsetzung}
\label{sec:4}
\subsection{Entwurf}

Um die Partikelsysteme umzusetzen wird der relativ junge, 
von Unity entwickelte Editor 'Visual Effects Graph' 
(kurz: VFX Graph) verwendet. VFX Graph ist ein nodesbasierter Editor, um schnell
dynamische Partikelsysteme zu erzeugen\footnote{https://unity.com/de/visual-effect-graph}.
Im Gegensatz zum älteren Shuriken-Partikelsystem von Unity werden die Partikel hier auf der GPU
simuliert, wodurch das System deutlich an Performance gewinnt und deutlich mehr Partikel zeichnen kann. 
Shuriken nimmt die Berechnungen auf der CPU vor\footnote{https://docs.unity3d.com/Manual/ChoosingYourParticleSystem.html}. 
VFX-Graph hat jedoch nur sehr begrenzte Möglichkeiten, 
was Physiksimulationen angeht. Es muss also ein System erstellt werden, welches ein möglichst 
realistisches Verhalten der Feuer- und Rauchpartikel gewährleistet ohne dabei auf tatsächlichen
physikalischen Berechnungen zu basieren.  



\subsection{Erstellung der Texturen}
\subsection{Unity (?)}
\subsection{Shader}

\subsection{Partikelsystem}





\newpage
