\section{Konzeption und Umsetzung}
\label{sec:4}
\subsection{Entwurf}

Da die Ergebnisse dieser Arbeit relevant für das KoViTReK Forschungsprojekt sein könnten und dieses in der 
Laufzeit- und Entwicklungsumgebung Unity\footnote{https://unity.com} implementiert werden soll, werden auch die Partikelsysteme in der selben 
Software entworfen. 

Um die Partikelsysteme umzusetzen wird der relativ junge, von Unity entwickelte Editor 'Visual Effects Graph' 
(kurz: VFX Graph) verwendet. VFX Graph ist ein nodesbasierter Editor, um schnell
dynamische und komplexe Partikelsysteme zu erzeugen\footnote{https://unity.com/de/visual-effect-graph}.
Im Gegensatz zum älteren Shuriken-Partikelsystem von Unity werden die Partikel hier auf der GPU
simuliert, wodurch das System deutlich an Performance gewinnt und deutlich mehr Partikel zeichnen kann. 
Shuriken nimmt die Berechnungen im Gegensatz zum VFX Graph auf der CPU vor\footnote{https://docs.unity3d.com/Manual/ChoosingYourParticleSystem.html}. 
Gerade für VR-Anwendungen bietet sich also dieses neue System an.
VFX-Graph hat jedoch nur sehr begrenzte Möglichkeiten, was Physiksimulationen und Kollisionen der Partikel angeht. 
Es muss also ein System erstellt werden, welches trotz der Einschränkungen ein möglichst realistisches Verhalten der Feuer- und Rauchpartikel 
gewährleistet.  


\subsection{Erstellung der Texturen}

Werden die Prozedural als Shader erstellt? 
-> POM müsste möglicherweise mit Blender/Houdini/Flipbooks erstellt werden
-> Raymarching kann man Prozedural generieren, da random Volume benötigt wird

\subsection{Shader}

\subsection{Partikelsystem}





\newpage
